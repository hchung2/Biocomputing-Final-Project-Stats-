 \documentclass[reqno, 11pt]{amsart}
\usepackage{pdfsync}
    \usepackage{amsmath} 
    \usepackage{amsthm}
\usepackage{helvet}
\usepackage{mathrsfs}
    \usepackage{graphicx} 
    \usepackage{caption}
    \usepackage{subcaption}
    \usepackage{float}
    \usepackage{hyperref}
    \hypersetup{
        colorlinks=true,
        linkcolor=blue,
        filecolor=magenta,      
        urlcolor=cyan,
    }
     
    \urlstyle{same}
    \usepackage{xcolor}
    \usepackage{cancel}
    \usepackage{listings}
    \usepackage[margin=1in]{geometry}
    \usepackage{amssymb}
    \usepackage{amsthm}
    \usepackage{breqn}
    \usepackage{enumerate}
    \usepackage{slashbox}
\usepackage{outlines}
\usepackage{enumitem}
\setenumerate[1]{label=\Roman*.}
\setenumerate[2]{label=\Alph*.}
\setenumerate[3]{label=\roman*.}
\setenumerate[4]{label=\alph*.}

    
    \newtheorem{theorem}{Theorem}[section]
	\newtheorem{axiom}[theorem]{Axiom}
    \newtheorem{lemma}[theorem]{Lemma}
    \newtheorem{prop}[theorem]{Proposition}
    \newtheorem{cor}[theorem]{Corollary}
    
    \theoremstyle{definition}
    \newtheorem{definition}[theorem]{Definition}
    \newtheorem{example}[theorem]{Example}
    \newtheorem{remark}[theorem]{Remark}
    \newtheorem{xca}[theorem]{Exercise}
    
    \numberwithin{equation}{section}
    
    %\theoremstyle{remark}
    %\newtheorem{remark}[theorem]{Remark}
    
    \newskip\aline \newskip\halfaline
\aline=12pt plus 1pt minus 1pt \halfaline=6pt plus 1pt minus 1pt
\def\skipaline{\vskip\aline}
\def\skiphalfaline{\vskip\halfaline}

\def\qedbox{$\rlap{$\sqcap$}\sqcup$}
\def\qed{\nobreak\hfill\penalty250 \hbox{}\nobreak\hfill\qedbox\skipaline}

\def\proofend{\eqno{\mbox{\qedbox}}}
    
 %blackboardbold
    \newcommand{\bE}{\mathbb{E}}
    \newcommand{\bP}{\mathbb{P}}
    \newcommand{\bR}{\mathbb{R}}
    
%mathscr

\newcommand{\eX}{\mathscr{X}}
\newcommand{\eY}{\mathscr{Y}}
    
 \begin{document}

\title{BioComputing Final Project}
\thanks{Started Novemver 27, 2017. Last revised \today}

\maketitle

\section{Things to Do}

\begin{outline}
\1 Analyze antibiotics.txt using an ANOVA-design linear model and likelihood ratio test.
	\2 Generate a plot to summarize the results of the experiment.
	\2 Write code to perform an ANOVA test comparing the three different treatments and the control.
	\2 Wirte a markdown document to summarize the results of the ANOVA test and make conclusions about hypotheses.
\1 Analyze sugar.txt using a regression design linear model and likelihood ratio test.
	\2 Generate a plot to summarize the results of the experiment.
	\2 Using a regression-design linear model and a likelihood ratio test, test for significance of treatment.
	\2 Write a markdown document to summarize the results and make conclusions about hypotheses.
\1 Perform a statistical power analysis comparing ANOVA and regression-designed experiments.
	\2 Assume a variable $y$ depends on an independent variable $x$ with slope $\beta_1=0.4$ and a $y$-intercept of $\beta_0=10$.  
	\2 Compare regression-design to a two-level ANOVA with $24$ experimental units.
		\3 Simulate $10$ random experiments with a regression design for $\sigma\in\{1,2,4,6,8,12,16,24\}$ for $x\in[0,50]$.
		\3 Simulate $10$ random experiments with a two-level ANOVA design  $\sigma\in\{1,2,4,6,8,12,16,24\}$ $x\in[0,50]$.
	\2 Compare regression-design to a four-level ANOVA with $24$ experimental units.
		\3 Simulate $10$ random experiments with a regression design for $\sigma\in\{1,2,4,6,8,12,16,24\}$ for $x\in[0,50]$.
		\3 Simulate $10$ random experiments with a four-level ANOVA design  $\sigma\in\{1,2,4,6,8,12,16,24\}$ $x\in[0,50]$.	
	\2 Compare regression-design to an eight-level ANOVA with $24$ experimental units.
		\3 Simulate $10$ random experiments with a regression design for $\sigma\in\{1,2,4,6,8,12,16,24\}$ for $x\in[0,50]$.
		\3 Simulate $10$ random experiments with an eight-level ANOVA design  $\sigma\in\{1,2,4,6,8,12,16,24\}$ $x\in[0,50]$.
	\2 Write a MarkDown file summarizing results
		\3 How did the ANOVA vs regression design perform?
			\4 Use the average $p$-values from the liklihood ratio tests across Monte-Carlo simulaitons as a metric of statistical power
		\3 Does the relative performance of the experimental designs depend on the number of levels in the ANOVA experiment?
\end{outline}



\end{document}